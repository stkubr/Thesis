\chapter{The QCD at low energies}
\label{chap:QCD_low}

\section{Introduction}
Quantum Chromodynamics (QCD) is widely approved to be the underlying theory of strong interactions. In a nutshell it is a local non-Abelian Jang-Mills gauge field theory with gauge group symmetry SU(3). The non-Abelian nature of its group renders itself on diagrammatic level as presence of self-interaction between glouns in addition to the quark-gluon interaction. 
\\

Been carriers of the "color", the corresponding to SU(3) gauge group charge, are not gauge independent objects themselves, and therefore are neither detectable nor exist as free, asymptotic states, though a certain relation of outgoing quark and hadron jet can be established at high energies due to quark-jet duality. States whose are detectable and had been seen in experiment are the bound states of quarks and presumably gluons. Without exaggeration, the study of such closed, confined objects is a sophisticated subject. And could have been even more if a quark would not carry, in addition to the color, the electric charge, described by gauge theory of Quantum Electodynamics (QED). This fact provides the possibility to test dynamical properties of bound states and to probe their quark substance by photonic scalpel. 
\\

Additionally to gauge group invariance, the Lagrangian of QCD posses several symmetries, some of whose are hidden. Of these, chiral symmetry and its breakdown are essential for the mass generation of bound states. 
\\

This chapter provide the introduction in the basic aspects of low energy QCD and serve as a starting point for the study of quark bound states, namely hadrons and their electromagnetic properties.



\section{The Quantum Chromodynamic field theory}
	\subsection{The QCD Lagrangian}
	Following the same ideas of the localization of the initially global gauge transformation as it was employed for U(1) group transformation in QED, as a starting point one can write down the Lagrangian of Dirac femionic field $q(x)$ with mass parameter $m$:
	\beqa
		\mathcal{L}_{fermions} = \bar{q}(i\dslash - m )q \;.
	\eeqa
	
	
	
	\subsection{The generating functional of QCD}
	\subsection{Fadeev-Popov ghosts} 
	\subsection{Gribov ambiguity}
	\subsection{Slavnov-Taylor identity}
	
\section{Symmetries of QCD}
	\subsection{Chiral symmetry}
	\subsection{Axial symmetry}
	
\section{Aspects of QCD}
	\subsection{Confinement}
	\subsection{Dynamical chiral symmetry breaking}
	\subsection{Running coupling of QCD}
	\subsection{Pion clouds}