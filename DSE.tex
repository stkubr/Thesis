\chapter{\DS equations}
\label{chap:dse}

\section{Quark DSE}
The Schwinger-Dyson equations are the equations of motion of the quantum field theory.
They are derived from the full generating functional of the quantum theory, Eq. (2.31)
for QCD. The starting point for the derivation of the Schwinger-Dyson equations is the
observation that the functional integral of a total derivative is zero:

We are only interested in the explicit derivation of the SDE for the quark propagator.
From this equation it will be clear what the set of SDE are. Since quark fields enter
additively into the Lagrangian of QCD, we focus on just one flavour, denoted by
Furthermore, we will not consider the ghost fields since these do not couple directly to
the quarks. The ghosts will enter into the quark SDE through the quark-gluon vertex,
and the gluon propagator. The derivation is in fact the same as that for the fermion
in QED, the only difference being the colour quantum number. Additionally, cubic and
quartic interactions between gauge bosons do not enter explicitly in our derivation. We
will consider thus the generating functional of QED.
To derive the SDE for the quark propagator, we apply Eq. (2.70) to the generating
functional of QED, with the functional derivative taken with respect to:

\section{Rainbow-Ladder truncation}
\section{Beyond RL approach}
\section{Numerical solution of the DSE}
	\subsection*{Integral equation}
	\subsection*{Properties of solution}
	\subsection*{Analytical and non- structure}