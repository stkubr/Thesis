\chapter*{Preface}
\label{chap:Intro}

Quantum Chromodynamics (QCD) is widely approved to be the underlying theory of strong interactions. In a nutshell it is a local non-Abelian Yang-Mills gauge field theory with gauge group symmetry SU(3). The non-Abelian nature of its group renders itself on diagrammatic level as presence of self-interaction between gluons in addition to the quark-gluon interaction \cite{Fritzsch1973365}. Since quarks are carriers of the "color", the $SU(3)$ gauge group charge and therefore not gauge independent objects themselves, neither detectable nor exist as free, asymptotic states. However a certain relation between the outgoing quark and the hadron jet can be established at high energies due to quark-jet duality. The states which are detectable and had been seen in experiment are the bound states of quarks and presumably gluons. The study of such closed, confined objects is a sophisticated subject and could have been even more if a quark would not carry, in addition to the color, the electric charge, described by gauge theory of Quantum Electodynamics (QED). This fact provides the possibility to test dynamical properties of bound states and to probe their inner quark substance by photonic scalpel, like Deep Inelastic Scattering. \\ 

The most noteworthy features of QCD are quark asymptotic freedom \cite{PhysRevLett.30.1343, Gross:1973ju}, dynamical chiral symmetry breaking \cite{PhysRev.122.345} and confinement \cite{Greensite201101}. Asymptotic freedom notes the fact that while at low energies the running coupling of QCD is significantly big, whether at high energies it becomes small enough for the perturbative theory to be applied. The dynamical chiral symmetry breaking (D$\chi$SB) occurs at low energies and plays the major role for QCD phenomenology. This effect has the immense value since it is responsible for the generation around 95\% of the mass of the visible universe. Confinement reflects the fact that although the elementary fields of the theory are quarks and gluons, they never appear in a experiment, eluding an experimentalist's eye since early searches in the lunar coat in 70-es. \\


%Bound states in QCD are composite color-scalar objects made of color-carrying particles. Starting from common two-body state $q \bar q$ like meson and three-body state $qqq$ like baryon, and ending with exotic not-yet-detected-but-possibly-existing states like tetraquarks $qq \bar q \bar q$, glueballs $GG$ and hybrids $qqG$. Due to usual form of propagator of massive particle $\frac{1}{p^2 + M^2}$ a bound state produce a pole in the scattering amplitude in the corresponding channel. For a composite bound state, the pole can not be generated by any finite sum of Feynman diagrams \cite{9780471353867}, but only by infinite series. However it is not possible in general, so instead we may consider to strive for an infinite sum of diagrams of a particular class, which are we assume to be dominant and crucial for a given process (i.e. all ladder diagrams). This can be archived by finding an appropriate integral equation, the solutions of which can be interpreted as the result of such particular summation.

One of the key features of QCD is existence of composite color-scalar objects made of color-carrying particles, such as quark-antiquark $q \bar q$ bound state called meson and three-quark $qqq$ bound state like baryon. After recent success of Babar, Belle and BES experimental facilities in discovering the XYZ charmonium bound states and charged states in bottomonium, the QCD spectroscopy became a intriguing topic. In addition to commonly known meson and baryons there may exist exotic colorless states like tetraquarks $qq \bar q \bar q$, glueballs $GG$ and hybrids $qqG$. Since the quarks in a bound state continuously exchange gluons on the Feynmann diagrams language this would require an infinite sum of diagrams. This cannot be archived in perturbative QCD, because this task requires enormous efforts and relatively small coupling constant. Additionally the bound states can enter into the play as virtual particles, being exchanged between the quarks, so that gives a rise to hadronic unquenching effects. Due to the pion being the lightest hadron, the pion exchange effect will be dominant among other hadronic exchange effects. Pion cloud effects are expected to play an important role in the low momentum behaviour of form factors and hadronic decay processes of baryons. \\



The fact that the most interesting part of QCD physics is hidden in low energy region and the lack of perturbative means to describe it, encouraged the development of various non-perturbative methods such as: quark models, Lattice QCD, $\chi$PT and functional methods. In this thesis we use the functional approach to QCD employing the quark \DSE in order to obtain non-perturbalive properties of quarks. Additionally, within meson 2-body \BSE and baryon 3-body Faddeev equations we provide a consistent description of QCD hadron phenomenology. \\ 


The thesis is organized as follows: in Chapter \ref{chap:QCD_low} we derive the QCD Lagrangian and review its basic properties and symmetries. In Chapter \ref{chap:DSE} we derive quark \DSE, consider the necessary truncations, conduct the required calculations and study the resulting solutions. The meson \BSE as well as Faddeev equations for baryons are derived and investigated in Chapter \ref{chap:BSE}. The arising solutions of meson BSE, its properties, mass spectra and Regge-trajectories for light and heavy quarks using the single gluon rainbow-ladder exchange are shown in Chapter \ref{chap:spectra}. The impact of pion cloud effect on meson mass spectra, Nucleon and Delta three body states as well as dynamical properties of pion, like the pion form factor, is studied in Chapter \ref{chap:pion}. Chapter \ref{chap:sum} summarizes the results and provides an outlook. \\ 

Part of the material in this thesis was reported in the following papers:
\newline
\newline
Sanchis-Alepuz, H. and Fischer, C. S. and Kubrak, S, \textit{Pion cloud effects on baryon masses}, \href{10.1016/j.physletb.2014.04.031}{{Phys.Lett. B733}} \newline
\newline
Fischer, C. S. and Kubrak, S. and Williams,R, \textit{Mass spectra and Regge trajectories of light mesons in the Bethe-Salpeter approach}, \href{http://10.1140/epja/i2014-14126-6}{{Eur.Phys.J. A50(2014)126}} \newline
\newline
Fischer, C. S. and Kubrak, S. and Williams,R, \textit{Spectra of heavy mesons in the Bethe-Salpeter approach}, \href{http://10.1140/epja/i2015-15010-7}{{Eur.Phys.J. A51(2015)1,10}}
