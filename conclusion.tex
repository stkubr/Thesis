\chapter{Summary and outlook}
\label{chap:sum}
\vspace{-1.0cm}
Some of the mysteries of QCD phenomenology can be face with the framework of the coupled quark \DSE, meson \BSE and baryon Faddeev equation, providing non-perturbative, continuum and Poincare invariant scientific approach. The research performed throughout this thesis is twofold. From one perspective we focus on the investigation of mass spectra for mesons with total spin quantum number $J=3$ and arising Regge-trajectory for natural parity states $J^{PC}=1^{--},2^{++},3^{--}$ within rainbow-ladder single gluon exchange model. The other findings are concerning the impact of the pion cloud effect on $J>2$ meson states, baryon masses, namely on Nucleon and Delta three-body bound states and meson dynamical properties like the pion form factor.\\
\vspace{-0.3cm}

For meson mass spectra studies we employ a simple interaction model, the effective gluon rainbow-ladder
approximation, which is known to represent only part of the complicated interaction pattern of quarks and gluons even for heavy quarks. However for the light quarks we obtained quantitatively reliable results for channels where only the contact part of the spin-spin interaction plays a role and channels dominated by the spin-orbit force, {\it i.e.} $J^{PC}= 2^{++}, 3^{--}$. The technical improvement that made available for the calculation mass spectra with quantum numbers $J=3$ allowed us to address the phenomena of Regge mass trajectory within DSE/BSE approach. Despite the fact that the rainbow-ladder approximation has clear deficiencies in the light quark sector we were able to obtain the Regge-trajectory behaviour for natural parity states $J^{PC}=1^{--},2^{++},3^{--}$ deviating from experimental data on the level of 5\%. In the heavy quark sector, where the rainbow-ladder approximation does particularly well, the agreement with the experimental states is much improved. We gave predictions for the tensor charmonia and bottomonia states, in particular for the $3^{--}$. We also gave results for $B_c$ states and quarkonia with exotic quantum numbers, although the accumulated errors in these channels due to deficiencies in the rainbow-ladder interaction may be sizeable.\\
\vspace{-0.3cm}

The another purpose of this thesis is to investigate the impact of the pion cloud effect on Nucleon and Delta three-body bound states masses and pion dynamical properties, specifically the pion electromagnetic form factor. This work complements the efforts in estimating the impact of hadronic unquenching effects, carried out in \cite{Fischer:2007ze,Fischer:2008sp,Fischer:2008wy}. We found substantial contributions of the pion cloud effects to the masses of the baryons of the order of 5-15 \%, depending on the parameters of the underlying quark-gluon interaction.
In addition, we found slight but significant changes in the structure of the baryons reflected in the relative contributions of their partial waves.
Concerning the pion form factor we found a slight deviation from gluon rainbow-ladder results in the range of intermediate momenta transferred, $0.75\; \text{GeV}^2 < Q^2 < 1.75\; \text{GeV}^2$. This deviation reflects the fact that with pion cloud included the pion form factor shows an extra substructure - the virtual pion cloud surrounding the pion quark core. However it is impossible to distinguish these two pictures and estimate the real impact of the pion cloud effect due to lack of experimental data and its accuracy. \\
\vspace{-0.3cm}

The plausible future directions of research would be the calculation of charmonium radiative decays: processes like $J/\psi \rightarrow \gamma \eta_c$, $\chi_{c0} \rightarrow \gamma J/\psi$, and etc. Also it is of the extreme interest is to find a robust way to extract the two-quark (pseudo-) potential out the meson Bethe-Salpeter equations with the given truncation. This can provide a better way to compare the quark potential model approaches to DSE/BSE framework, since this would let us clear understand the impact of the employed truncation on the spin-spin and spin-orbit parts of the two-quark potential. The another direction would be to extend the employed pion cloud framework to baryon form factor calculations.

